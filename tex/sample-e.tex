\documentclass{deime}
%\usepackage[dvipdfm]{graphicx}
%\usepackage{latexsym}
%\usepackage{txfonts}
%\usepackage[fleqn]{amsmath}
%\usepackage[psamsfonts]{amssymb}
%\usepackage[deluxe]{otf}

% 印刷位置調整 %
% 必要に応じて値を変更してください.
\hoffset -10mm % <-- 左に 10mm 移動
\voffset -10mm % <-- 上に 10mm 移動

\newcommand{\AmSLaTeX}{%
 $\mathcal A$\lower.4ex\hbox{$\!\mathcal M\!$}$\mathcal S$-\LaTeX}
\newcommand{\PS}{{\scshape Post\-Script}}
\def\BibTeX{{\rmfamily B\kern-.05em{\scshape i\kern-.025em b}\kern-.08em
 T\kern-.1667em\lower.7ex\hbox{E}\kern-.125em X}}

\papernumber{DEIM Forum 2014 XX-Y}

\etitle{DEIM Forum 2014 Class File }
\esubtitle{Subtitle}
\authorlist{%
 \authorentry[kaoru@lafore.ac.jp]{内海 薫}{Kaoru UTSUMI}{Shuzenji,Teito}% 
 \authorentry[manabu@royal-u.ac.jp]{湯川 学}{Manabu YUKAWA}{Teito}% 
 \authorentry[kusanagi@gmail.com]{草薙 俊平}{Shunpei KUSANAGI}{Kyoto}% 
}
\affiliate[Shuzenji]{修善寺大学情報学部\hskip1zw
  〒410--2415 静岡県伊豆市大平1529}
 {Faculty of Information Science and Engineering,
  Shuzenji University\\
  1529 Ohira, Izu, Shizuoka,
  410--2415 Japan}
\affiliate[Teito]{帝都大学理工学部物理学科\hskip1zw
  〒100--0001 東京都千代田区千代田1番1号}
 {College of Science and Technology, Royal
   University of Tokyo\\
  1--1 Chiyoda, Chiyoda, Tokyo,
  100--0001 Japan}
\affiliate[Kyoto]{帝都大学工学部情報工学科\hskip1zw
  〒602--0881 京都府京都市上京区京都御苑3}
 {Faculty of Engineering, Royal University of Kyoto\\
  Kyoto-Gyoen 3, Kamigyo, Kyoto
  602--0881 Japan}

%\MailAddress{$\dagger$hanako@deim.ac.jp,
% $\dagger\dagger$\{taro,jiro\}@jforum.co.jp}

\begin{document}
\pagestyle{empty}
\begin{eabstract}
Paper format for DEIM Forum 2014 Proceedings.
\end{eabstract}
\begin{ekeyword}
p\LaTeXe\ class file, typesetting
\end{ekeyword}
\maketitle

\section{タイトル・概要に関して}

1ページ目上部には,タイトル,発表者氏名,所属,住所,メールアドレス,キーワードの和文と英文及びあらまし(和文300字程度,英文100語程度)を,それぞれ記述してください.
なお、和文論文については英文タイトル,アブストラクト等は削除して頂いて構いません。
下記のコマンドで講演番号を挿入して下さい.
\begin{verbatim}
 \papernumber{DEIM Forum 2014 XX-Y}
\end{verbatim}
XXはセッション番号(例:1A, 3B),Yはセッション内での発表順(1, 2, ...)です.
番号についてはプログラムをご覧ください.
なお,プログラム決定前の初回投稿時にはXX-Yの部分の記入は不要です.

\section{原稿提出枚数}

所定のページ数(4~8ページ)を厳守してください.
Ph.Dセッション投稿者は8ページを推奨します.

\section{原稿の書き方}

原稿のスタイルは,A4サイズで,9ポイントのフォントを使用し,2段組み,シングルスペースとして下さい.

\vspace{30mm}

\begin{thebibliography}{99}
\bibitem{Minmei2009}
大河内民明丸,「分子核構造その理論」,民明書房

\bibitem{Tanaka2011}
Lilis Aadam, Lilin Tabris, and Tohji Suzuhara ``SKATER'SWALTZ: Good Presentation for DEIM'', Proc. of ACM ROBOT, pp. 10-22, 2010.
\end{thebibliography}


\end{document}
